% Template für Funktionsdokumentation


% Hier den Namen der Funktion eintrage
\Reference{madForward}


% Kurzbeschreibung (1 Satz)
\begin{Purpose}
Calculates the scaled Taylor coefficients of the image path.
\end{Purpose}


% Funktionsdeklaration mit Argumenten und Rückgabewerten
\begin{Synopsis}
Z = madForward(TapeId, d, keep, X)
\end{Synopsis}


% Detaillierte Beschreibung
\begin{Description}
This function calculates the scaled Taylor coefficients $\mathbf{z}_i$ ($i=0,\ldots,d)$ of the image path 
\begin{equation*}
\mathbf{z}(t) = \mathbf{F}(\mathbf{x}(t)) = \mathbf{z}_0 + \mathbf{z}_1 t + \mathbf{z}_2 t^2 + \ldots + \mathbf{z}_d t^d + \mathcal{O}(t^d)
\end{equation*}
of the function $\mathbf{F}$ represented by the tape with the tape number \texttt{TapeId}. The \texttt{n} $\times$ \texttt{d} matrix \texttt{X} contains the Taylor coefficients $\mathbf{x}_i$ ($i=0,\ldots,d$) of the path 
\begin{equation*}
\mathbf{x}(t) = \mathbf{x}_0 + \mathbf{x}_1 t \mathbf{x}_2 t^2 + \ldots + \mathbf{x}_d t^d + \mathcal{O}(t^d)
\end{equation*}
with \texttt{n} the number of independent variables of the function and \texttt{d} the number of derivatives of $\mathbf{x}(t)$. The flag \texttt{keep} determines how many derivatives are internally stored for the use of the reverse mode. The following must hold: $1 \leq \mathtt{keep} \leq \mathtt{d}+1$. The function returns a $\mathtt{m} \times \mathtt{d}+1$ matrix containing the Taylor coefficients $\mathbf{z}_i$ ($i=0,\ldots,d$) of the image path with \texttt{m} the number of dependent variables.

Keep in mind that the Taylor coefficients of the paths are defined as follows:
\begin{equation*}
\mathbf{x}_k = \frac{1}{k!}\pdiff{^k}{t^k}\mathbf{x}(t) \qquad \mathbf{z}_k = \frac{1}{k!}\pdiff{^k}{t^k}\mathbf{z}(t) \qquad k = 0, \ldots, d.
\end{equation*}
\end{Description}


% Aufrufbeispiele mit Ergebnissen
\begin{Examples}
It is assumed that the function
\begin{equation*}
\mathbf{y} = 
\begin{pmatrix}
x_1^2x_2 + x_1\cos(x_2)\\
x_2^2\sin(x_1) + x_2x_1^2
\end{pmatrix}
\end{equation*}
is represented by the tape with the number \texttt{TapeId}. If one has $x_1(0) = 3, \dot x_1(0) = -3, \ddot x_1(0) = -6$ and $x_2(0) = 4, \dot x_1(0) = 4, \ddot x_1(0) = 8$ and one wants to calculate $\mathbf{z}_0, \mathbf{z}_1, \mathbf{z}_2$ one has to do the following:
\begin{verbatim}
X = [3 4; -3 4; -3 4]';
Z = madForward(TapeId, 2, 2, X)
Z = 34.0391 -24.9574 -54.3516
    38.2579  16.0355  67.1720
\end{verbatim}
\end{Examples}


% Einschränkungen, die bei der Nutzung zu berücksichtigen sind
%\begin{Limitations}
%\end{Limitations}


% Verweis auf andere Funktionen, die für die Nutzung dieser
% wichtig sein könnten
\begin{Seealso}
madTapeCreate, madTapeOpen, madTapeClose, madReverse
\end{Seealso}



%%% Local Variables: 
%%% mode: latex
%%% TeX-master: "AdolC4MatlabDoc"
%%% End: 
