% Dateiname: mathcmd.tex
%
% Jan Winkler
%
% MAkros f�r mathematische Ausdr�cke, Zeichen etc.


% Befehl                 Kurzbeschreibung
% ---------------------------------------------------

% Gleichung
\newcommand{\Gl}[1]{Gl.\hspace{-0.5ex} \eqref{#1}}

% Differntial
\newcommand{\df}{\mathrm{d}}

% Partielle Ableitung  #1: Zaehler, #2: Nenner
\newcommand{\pdiff}[2]
{
 \frac{\partial #1}{\partial #2}
}

% Partielle Ableitung, in Zeile geschrieben
\newcommand{\pdiffz}[2]
{
 \partial #1/\partial #2
}

% Ableitung  #1: Zaehler, #2: Nenner
\newcommand{\diff}[2]
{
 \frac{\mathrm{d} #1}{\mathrm{d} #2}
}

% Ableitung, in Zeile geschrieben
\newcommand{\diffz}[2]
{
 \mathrm{d} #1/\mathrm{d} #2
}

% n.te Ableitung in Zeile geschrieben:
\newcommand{\nderiv}[2]
{
   #1^{(#2)}
}

% Norm
\newcommand{\norm}[1]
{
 \left|\left|#1\right|\right|
}

% Gradient
\newcommand{\grad}
{
 \, \mathrm{grad}
}


% Divergenz
\newcommand{\diver}
{
 \, \mathrm{div}\,
}

% Modulo
\newcommand{\modulo}
{
 \, \mathrm{mod}\,
}


% Biot-Zahl
\newcommand{\Bi}
{
 \, \mathrm{Bi}
}


% Reynoldszahl
\newcommand{\Rey}
{
 \, \mathrm{Re}
}


% Peclet-Zahl
\newcommand{\Pec}
{
 \, \mathrm{Pe }
}


% Prandtl-Zahl
\newcommand{\Pra}
{
 \, \mathrm{Pr}
}

% C  MENGE KOMPLEXER ZAHLEN
\newcommand{\C}{{\sf C\hspace{-1.1ex}I}}

% N  MENGE NATUERLICHER ZAHLEN
\newcommand{\N}{{\sf I\hspace{-0.25ex}N}}

% R  MENGE REELLER ZAHLEN
\newcommand{\R}{{\sf I\hspace{-0.25ex}R}}
\newcommand{\Rp}{{\sf I\hspace{-0.25ex}R^+}}
\newcommand{\Rm}{{\sf I\hspace{-0.25ex}R^-}}


% Z  MENGE GANZER ZAHLEN
\newcommand{\Z}{{\sf Z\hspace{-0.9ex}Z}}

% Punkt am Definitionsende
\newcommand{\defnend}{\hfill \rule{2mm}{2mm}}

% Punkt am Formelende
\newcommand{\eqnend}{\\[-1.2cm] \rightline{\rule{2mm}{2mm}}}

% fette mathematische Buchstaben ( Quelle: Martin ):
\def\Mbf#1{\mbox{\boldmath$#1$}}


%%% Local Variables: 
%%% mode: plain-tex
%%% TeX-master: t
%%% End: 
