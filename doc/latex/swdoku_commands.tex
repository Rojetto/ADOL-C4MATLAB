% DEMClasses Softwaredocumentation
% Jan Winkler
% Robert Bosch GmbH, Stuttgart, FV/FLI
% Februar 2000
%
% Variierte Fassung zu befehle.tex von 
% Softwaredokumentation QuaMo-Toolbox
% AB Regelungstechnik, TUHH
% 20.12.1996

% Englische Fassung

%%%%%%%%%%%%%%%%%%%%%%%%
%%%   Extrabefehle   %%%
%%%%%%%%%%%%%%%%%%%%%%%%

% DF
\newcommand{\markieren}[1]
% Markieren eines Wortes mit \tt, nicht benutzt
{ {\tt #1} }


%%%%%%%%%%%%%%%%%%%%%%%%%%%%%%%%%%%%%%%%%%%
%   Gleichungsnummer in runden Klammern   %
%%%%%%%%%%%%%%%%%%%%%%%%%%%%%%%%%%%%%%%%%%%

\newcommand{\eqnref}[1]{(\ref{#1})}


%%%%%%%%%%%%%%%%%%%%%%%%%%%%%%%%%%%%%%%%%%%%%%%%%%%%%%%%%%%%%%%%%%


%%%%%%%%%%%%%%%%%%%%%%%%%%%%%%%%%%%%%%%%%%%%
%   Befehle f�r den endg�ltigen Ausdruck   %
%%%%%%%%%%%%%%%%%%%%%%%%%%%%%%%%%%%%%%%%%%%%

\newcommand{\mylabel}[1]{\label{#1}}
\newcommand{\myeqnref}[1]{\eqnref{#1}}
\newcommand{\myref}[1]{\ref{#1}}
\newcommand{\mypageref}[1]{\pageref{#1}}
\newcommand{\mybibitem}[1]{\bibitem{#1}}
\newcommand{\mycite}[1]{\cite{#1}}
\newcommand{\myindex}[1]{\index{#1}}


%%%%%%%%%%%%%%%%%%%%%%%%%%%%%%%%%%
%   Befehle f�r Probeausdrucke   %
%%%%%%%%%%%%%%%%%%%%%%%%%%%%%%%%%%

%\newcommand{\mylabel}[1]{\label{#1}
%                         \mbox{$|{\sl label = #1}|$}}
%newcommand{\myeqnref}[1]{(\ref{#1}
%                          \mbox{$|{\sl label = #1}|$})}
%\newcommand{\myref}[1]{\ref{#1}
%                      \mbox{$|{\sl label = #1}|$}}
%\newcommand{\mypageref}[1]{\pageref{#1}
%                      \mbox{$|{\sl label = #1}|$}}
%\newcommand{\mybibitem}[1]{\bibitem{#1}
%                           \mbox{$|{\sl bibitem = #1}|$}}
%\newcommand{\mycite}[1]{\cite{#1}
%                        \mbox{$|{\sl bibitem = #1}|$}}
%\newcommand{\myindex}[1]{\index{#1}
%                        \mbox{$|${\sl index = #1}$|$}}

%%%%%%%%%%%%%%%%%%%%%%%%%%%%%%%%%%%%%%%%%%%%%%%%%%%%%%%%%%%%%%%%%%


%%%%%%%%%%%%%%
%   Absatz   %
%%%%%%%%%%%%%%

\newcommand{\newpar}{\\[2ex]}
\newcommand{\newp}{\\[1ex]}


%%%%%%%%%%%%%%%%%%%%%%%%%%
%   Gliederungsbefehle   %
%%%%%%%%%%%%%%%%%%%%%%%%%%
% Abschnitt



\newcommand{\mypart}[1]{
            \vspace*{5ex}
            \part{#1}
            %\vspace{5ex}
}

      
% Kapitel (Numerierung *)

\newcommand{\mysection}[1]{
            \setcounter{mypage}{1}
            \vspace*{5ex}
            \section{#1}
            \vspace{5ex}}


% Unterkapitel (Numerierung *.*);

\setcounter{subsection}{0}

\newcommand{\mysubsection}[1]{
            % \clearpage
            \subsection{#1}
            \vspace{2ex}
            \unitlength1cm
            % \begin{picture}(12,0)
            %               \put(-4,0){\line(1,0){16}}
            % \end{picture}\vspace{2ex}
            }


% Unterkapitel (ohne Numerierung)

\newcommand{\mysubsubsection}[1]{
            \subsubsection*{#1}
            \addcontentsline{toc}{subsubsection}{\hspace{0.5cm} #1}}


% �berschrift f�r Functionen im Reference-Teil;
% erzeugt Linie auf dem Seitenkopf, Eintrag in das
% Inhaltsverzeichnis und Indexregister

\newcommand{\Reference}[1]{
            \subsection*{#1}
            \addcontentsline{tor}{subsection}{#1}
            % \markright{#1}% \hfil}{\leftmark \hfil #1}
            \markright{Member Function - #1}% \hfil}{\leftmark \hfil #1}
            \vspace{-3ex}
            \unitlength1cm
            \begin{picture}(12,0)
                           \put(-4,0){\line(1,0){16}}
            \end{picture}

            \myindex{#1}}


% �berschrift f�r Bl�cke im Reference-Teil;
% erzeugt Linie auf dem Seitenkopf, Eintrag in das
% Inhaltsverzeichnis und Indexregister

\newcommand{\Blocks}[1]{
            \subsection*{#1}
            \addcontentsline{tob}{subsection}{#1}
            \markright{Blocks - #1}% \hfil}{\leftmark \hfil #1}
            \vspace{-3ex}
            \unitlength1cm
            \begin{picture}(12,0)
                           \put(-4,0){\line(1,0){16}}
            \end{picture}

            \myindex{#1}}





% �berschrift f�r Datentypen im Reference-Teil;
% erzeugt Linie auf dem Seitenkopf, Eintrag in das
% Inhaltsverzeichnis und Indexregister

\newcommand{\Datatype}[1]{
            \subsection*{#1}
            \addcontentsline{tod}{subsection}{#1}
            % \markright{Datatype - #1}% \hfil}{\leftmark \hfil #1}
            \markright{Member Variable - #1}% \hfil}{\leftmark \hfil #1}
            \vspace{-3ex}
            \unitlength1cm
            \begin{picture}(12,0)
                           \put(-4,0){\line(1,0){16}}
            \end{picture}

            \myindex{#1}}



%%%%%%%%%%%%%%%%%%%%%%%%%%%%%%
%   Umgebung f�r Beispiele   %
%%%%%%%%%%%%%%%%%%%%%%%%%%%%%%

\newenvironment{Example}{\begin{quote}{\bf Example}\\}{\end{quote}}


%%%%%%%%%%%%%%%%%%%%%%%%%%%%%%%%%%%%%%%%%%%%%%%%%
%   Gliederungsbefehle f�r den Reference-Teil   %
%%%%%%%%%%%%%%%%%%%%%%%%%%%%%%%%%%%%%%%%%%%%%%%%%

\newenvironment{Purpose}{\paragraph{Purpose}}{}

\newenvironment{Icon}[2]{
            \vspace{0.5cm}
            \hspace*{-4cm}
            \begin{minipage}[t]{3.8cm}
              \hphantom{phantom}
              \vspace{-0.4cm}
              \leavevmode
              \epsfxsize=2.2cm
              \epsffile{#1}
            \end{minipage}
            \begin{minipage}[t]{12cm}
              #2
            }{\end{minipage}}

\newenvironment{Synopsis}{\paragraph{Synopsis} \tt}{}

\newenvironment{Description}{\paragraph{Description}}{}

\newenvironment{Examples}{\paragraph{Examples}}{}

\newenvironment{Algorithm}{\paragraph{Algorithm}}{}

\newenvironment{Limitations}{\paragraph{Limitations}}{}

\newenvironment{Diagnostics}{\paragraph{Diagnostics}}{}

\newenvironment{Subroutines}{\paragraph{Subroutines} \tt}{}

\newenvironment{Seealso}{\paragraph{See Also} \tt}{}

\newenvironment{References}{\paragraph{References}}{}

\newenvironment{WA}{\paragraph{Write Access}}{}

\newenvironment{RA}{\paragraph{Read Access}}{}



%%%%%%%%%%%%%%%%%%%%%%%%%%%%%%%%
%   Ausgabe von Programmtext   %
%%%%%%%%%%%%%%%%%%%%%%%%%%%%%%%%

\newenvironment{Progtext}{\begin{quote} 
                          \renewcommand{\baselinestretch}{0.8}}{\end{quote}}



%%%%%%%%%%%%%%%%
%   Tabellen   %
%%%%%%%%%%%%%%%%

% Tabelle als Verzeichnis der Funktionen

\newenvironment{Mtabular}[2]{
     \renewcommand{\arraystretch}{1.4}
     \hspace*{-1.2cm}
     \hfill
     \begin{tabular}{|p{2.6cm} p{9.6cm}|}
       \hline
       \multicolumn{2}{|c|}{\rule[-3mm]{0mm}{8mm}\large\bf #1}\\
       \hline
       #2
       \hline}{
     \end{tabular}
     \vspace{0.8cm}}


% Tabelle als Verzeichnis der Bl�cke

\newenvironment{Btabular}[2]{
     \renewcommand{\arraystretch}{1.4}
     \hspace*{-1.2cm}
     \hfill
     \begin{tabular}{|p{4.6cm} p{7.6cm}|}
       \hline
       \multicolumn{2}{|c|}{\rule[-3mm]{0mm}{8mm}\large\bf #1}\\
       \hline
       #2
       \hline}{
     \end{tabular}
     \vspace{1cm}}


% Tabelle f�r die Parameter der Simulink-Bl�cke

\newenvironment{Parameters}{
     \paragraph{Parameters}
     \renewcommand{\arraystretch}{1.4}
     \vspace{1ex}
     \begin{tabular}{|l @{\hspace{3em}} l @{\hspace{1em}}|}
       \hline}{
       \hline     
     \end{tabular}
     }



%%%%%%%%%%%%%%%%%%%%%%%%%%%%%%%%%%%%%%
%   fette mathemtische Buchstaben   %%
%%%%%%%%%%%%%%%%%%%%%%%%%%%%%%%%%%%%%%

\def\Mbf#1{\mbox{\boldmath$#1$}}


%%%%%%%%%%%%%%%%%%%%%
%   Zahlenmengen   %%
%%%%%%%%%%%%%%%%%%%%%

%\newcommand{\C}{{\sf C\hspace{-1.1ex}I}}
%\newcommand{\N}{{\sf I\hspace{-0.25ex}N}}
%\newcommand{\R}{{\sf I\hspace{-0.25ex}R}}
%\newcommand{\Z}{{\sf Z\hspace{-0.9ex}Z}}


%%%%%%%%%%%%%%%%%%%%%%%%%%%%%%%%%%%%%%%%%%%%%%%%%%%%%%%%%%%%%%%%%%
%%%%%%%%%%%%%%%%%%%%%%%%%%%%%%%%%%%%%%%%%%%%%%%%%%%%%%%%%%%%%%%%%%


%%%%%%%%%%%%%%%%%%%%%%%%%%%%%%%%%%%%%%%%%%%
%   Befehle zum Einbinden von EPS-Files   %
%%%%%%%%%%%%%%%%%%%%%%%%%%%%%%%%%%%%%%%%%%%


%%%%%%%%%%%%%%%%%%%%%%%%%%%%%%%%%%%%%%%%%%%%%%%%%%%%%%%%%%%%
%                                                          %
%   Umdefinieren von \epsfsize{} laut DVIPS-Manual         %
%                                                          %
%   Die Default-Definition in epsf.sty lautet:             %
%   \def\epsfsize#1#2{\epsfxsize}                          %
%                                                          %
%   1. Falls nat�rliche Bildbreite gr��er als Textbreite,  %
%   wird die Bildbreite auf die Textbreite begrenzt.       %
%   Die Skalierung durch \epsfxsize bzw.                   %
%   \epsfysize (s. EpsbildW, EpsbildH) wird durch diese    %
%   Umdefinition unm�glich.                                %
%   \def\epsfsize#1#2{                                     %
%   \ifdim #1 > \textwidth \textwidth \else #1 \fi         %
%   }                                                      %
%                                                          %
%   2. Falls \epsfxsize gesetzt sicherstellen, da�         %
%   Bildbreite (\epsfxsize) <= Textbreite. Falls           %
%   \epsfxsize nicht gesetzt (=0) sicherstellen, da�       %
%   Bildbreite (#1=nat�rliche Bildbreite) <= Textbreite.   %
%   ! Bei Verwendung von 'EpsbildH' kann das               %
%   unerwartete Effekte geben !                            %
%   \def\epsfsize#1#2{                                     %
%     \ifdim \epsfxsize = 0cm                              %
%        \ifdim #1 > \textwidth \textwidth                 %
%        \else #1 \fi                                      %
%     \else                                                %
%        \ifdim \epsfxsize > \textwidth \textwidth         %
%        \else \epsfxsize \fi                              %
%     \fi                                                  %
%   }                                                      %
%                                                          %
%%%%%%%%%%%%%%%%%%%%%%%%%%%%%%%%%%%%%%%%%%%%%%%%%%%%%%%%%%%%


%%%%%%%%%%%%%%%%%%%%%%%%%%%%%%%%%%%%%%%%%%%%%%%
%   Befehl zum Einbinden von EPS-Bildern      %
%   Aufruf: Epsbild{eps_file}{wohin}          %
%           {bildbeschriftung}{bezugsmarke}   %
%%%%%%%%%%%%%%%%%%%%%%%%%%%%%%%%%%%%%%%%%%%%%%%

\newcommand{\Epsbild}[4]{
            \begin{figure}[#2]
              \centering
              \leavevmode
              \epsffile{#1}
              \protect\caption{#3}
              \protect\mylabel{#4}
            \end{figure}
           }


%%%%%%%%%%%%%%%%%%%%%%%%%%%%%%%%%%%%%%%%%%%%%%%%%%%%%%%%%%%%%%%%%
%   Befehl zum Einbinden von EPS-Bildern mit Vorgabe der H�he   %
%   und �bergabe von Positionierungsparametern                  %
%   Aufruf: EpsbildH{eps_file}{wohin}{hoehe}                    %
%           {bildbeschriftung}{bezugsmarke}.                    %
%%%%%%%%%%%%%%%%%%%%%%%%%%%%%%%%%%%%%%%%%%%%%%%%%%%%%%%%%%%%%%%%%

\newcommand{\EpsbildH}[5]{
            \begin{figure}[#2]
              \centering
              \leavevmode
              \epsfysize=#3
              \epsffile{#1}
              \protect\caption{#4}
              \protect\mylabel{#5}
            \end{figure}
           }


%%%%%%%%%%%%%%%%%%%%%%%%%%%%%%%%%%%%%%%%%%%%%%%%%%%%%%%%%%%%%
%   Befehl zum Einbinden von EPS-Bildern mit Vorgabe        %
%   der Breite und �bergabe von Positionierungsparametern   %
%   Aufruf: EpsbildW{eps_file}{wohin}{breite}               %
%           {bildbeschriftung}{bezugsmarke}.                %
%%%%%%%%%%%%%%%%%%%%%%%%%%%%%%%%%%%%%%%%%%%%%%%%%%%%%%%%%%%%%

\newcommand{\EpsbildW}[5]{
            \begin{figure}[#2]
              \centering
              \leavevmode
              \epsfxsize=#3
              \epsffile{#1}
              \protect\caption{#4}
              \protect\mylabel{#5}
            \end{figure}
           }


%%%%%%%%%%%%%%%%%%%%%%%%%%%%%%%%%%%%%%%%%
%   Befehl wie EpsbildW                 %
%   zuz�glich Erzeugung eines Rahmens   %
%%%%%%%%%%%%%%%%%%%%%%%%%%%%%%%%%%%%%%%%%

\newcommand{\EpsbildWR}[5]{
            \begin{figure}[#2]
              \centering
              \leavevmode
              \epsfxsize=#3
              \fbox{\epsffile{#1}}
              \protect\caption{#4}
              \protect\mylabel{#5}
            \end{figure}
           }


%%%%%%%%%%%%%%%%%%%%%%%%%%%%%%%%%%%
%   Befehl wie EpsbildW           %
%   ohne Unterschrift und Label   %
%%%%%%%%%%%%%%%%%%%%%%%%%%%%%%%%%%%

\newcommand{\EpsbildWo}[3]{
            \begin{figure}[#2]
              \centering
              \leavevmode
              \epsfxsize=#3
              \epsffile{#1}
            \end{figure}
           }


